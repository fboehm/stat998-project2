\documentclass[11pt,]{article}
\usepackage{lmodern}
\usepackage{amssymb,amsmath}
\usepackage{ifxetex,ifluatex}
\usepackage{fixltx2e} % provides \textsubscript
\ifnum 0\ifxetex 1\fi\ifluatex 1\fi=0 % if pdftex
  \usepackage[T1]{fontenc}
  \usepackage[utf8]{inputenc}
\else % if luatex or xelatex
  \ifxetex
    \usepackage{mathspec}
    \usepackage{xltxtra,xunicode}
  \else
    \usepackage{fontspec}
  \fi
  \defaultfontfeatures{Mapping=tex-text,Scale=MatchLowercase}
  \newcommand{\euro}{€}
\fi
% use upquote if available, for straight quotes in verbatim environments
\IfFileExists{upquote.sty}{\usepackage{upquote}}{}
% use microtype if available
\IfFileExists{microtype.sty}{%
\usepackage{microtype}
\UseMicrotypeSet[protrusion]{basicmath} % disable protrusion for tt fonts
}{}
\usepackage[margin=2cm]{geometry}
\ifxetex
  \usepackage[setpagesize=false, % page size defined by xetex
              unicode=false, % unicode breaks when used with xetex
              xetex]{hyperref}
\else
  \usepackage[unicode=true]{hyperref}
\fi
\hypersetup{breaklinks=true,
            bookmarks=true,
            pdfauthor={Fred Boehm, Statistics 998},
            pdftitle={Identifying myocardial infarction risk factors in the Wisconsin Longitudinal Survey},
            colorlinks=true,
            citecolor=blue,
            urlcolor=blue,
            linkcolor=magenta,
            pdfborder={0 0 0}}
\urlstyle{same}  % don't use monospace font for urls
\setlength{\parindent}{0pt}
\setlength{\parskip}{6pt plus 2pt minus 1pt}
\setlength{\emergencystretch}{3em}  % prevent overfull lines
\setcounter{secnumdepth}{0}

%%% Use protect on footnotes to avoid problems with footnotes in titles
\let\rmarkdownfootnote\footnote%
\def\footnote{\protect\rmarkdownfootnote}

%%% Change title format to be more compact
\usepackage{titling}
\setlength{\droptitle}{-2em}
  \title{Identifying myocardial infarction risk factors in the Wisconsin
Longitudinal Survey}
  \pretitle{\vspace{\droptitle}\centering\huge}
  \posttitle{\par}
  \author{Fred Boehm, Statistics 998}
  \preauthor{\centering\large\emph}
  \postauthor{\par}
  \predate{\centering\large\emph}
  \postdate{\par}
  \date{March 26, 2015}


\usepackage[hypcap]{caption}
\usepackage{subcaption}
\usepackage[colorinlistoftodos=true]{todonotes}
\usepackage{dcolumn}
\usepackage{booktabs}


\begin{document}

\maketitle


\listoftodos

\subsection{Abstract}\label{abstract}

\subsection{Introduction}\label{introduction}

Coronary artery disease (CAD) is a leading cause of death in the United
States and much of North America and Europe. In 2011, one American died
of CAD every 40 seconds, on average, and 155,000 of those deaths were
people aged less than 65 years (Mozaffarian et al. 2015). One
manifestation of CAD is a myocardial infarction (MI), which is also
called a ``heart attack''. A MI results from a clot in a coronary artery
that diminishes blood flow to the heart muscle, or myocardium. If blood
flow disruption persists for a sufficiently long time, the muscle may
die, or infarct. The irreparable dead heart muscle diminishes the
overall ability of the heart to pump blood. Severe MIs may lead to a
patient's death.

Epidemiologists have identified modifiable and non-modifiable risk
factors that contribute to CAD risk. Smoking is among the strongest
modifiable risk factors, and is thought to elevate CAD risk by
triggering elevations in inflammatory molecules in the bloodstream.
\_\_\_\_
\todo[inline, color = red]{What is mechanism for smoking causing CAD?}.
Diabetes mellitus and hypertension (systolic or diastolic) are typically
considered non-modifiable risk factors, although their contribution to
CAD risk may be reduced in patients who undertake dramatic lifestyle
interventions, such as exercise programs and diet with weight loss.
Non-modifiable risk factors include age, a family history of CAD and
presence of certain genetic variants
\todo[inline, color = pink]{what are other known risk factors?}.

Our collaborators at the Wisconsin Longitudinal Study (WLS) have
undertaken an investigation on a subset of WLS participants with the
goal of identifying CAD risk factors in the WLS study population. The
ultimate goal of this project is to develop an intervention program to
reduce CAD morbidity and mortality in Wisconsin. The investigators would
like to extend such an intervention program to Wisconsin residents who
are not WLS subjects. Our goal in this report is to identify risk
factors for MI among WLS participants.

\subsection{Study design}\label{study-design}

The Wisconsin Longitudinal Study (WLS) is a long-term study of a random
sample of 10,317 men and women who graduated from Wisconsin high schools
in 1957. According to the WLS website ``WLS provides an opportunity to
study the life course, intergenerational transfers and relationships,
family functioning, physical and mental health and well-being, and
morbidity and mortality from late adolescence through
2011.''(``Wisconsin Longitudinal Study'' 2015)

\todo[inline, color = blue]{need more info on WLS?}

Our collaborators collected data from the original respondents or their
parents in 1957, 1964, 1975, 1992, 2004, and 2011; from a selected
sibling in 1977, 1994, 2005, and 2011; from the spouse of the original
respondent in 2004; from the spouse of the selected sibling in 2006; and
from widow(er)s of the graduates and siblings in 2006.

\subsection{Data description}\label{data-description}

Our collaborators shared with us a data set that contains records for
19095 individuals (including original subjects and siblings) with 310
variables per subject. 2209 subjects responded (with yes or no) to the
2011 question of whether they had ever had a heart attack.

\subsection{Exploratory data analyses}\label{exploratory-data-analyses}

\subsection{Statistical modeling}\label{statistical-modeling}

We used statistical modeling to try to identify covariates that
associated with six distinct outcomes: 1) HA2004, 2) HA2011, 3) DOC2004,
4) DOC2011, 5) new self-reported heart attacks (from 2004 to 2011) and
6) new heart attack per doctor's report (from 2004 to 2011). For a
subject to qualify as a ``new'' self-reported heart attack, they must
have responded ``No'' in 2004 and ``Yes'' in 2011. Analogous definition
applies for ``new'' doctor-reported heart attack.

We found that HA2004 had 1695 non-missing responders (with 665
responding ``Yes'' and 1030 responding ``No''). Further response counts
are provided in

\subsubsection{Statistical modeling with Framingham study
predictors}\label{statistical-modeling-with-framingham-study-predictors}

We identified those variables in the WLS that closely match those in the
Framingham study (D'Agostino et al. 2008) ( Table \ref{tab:fram2wls}).
It's important to note that the Framingham study used survival analysis
methods, including Cox proportional hazards regression, to identify risk
factors for a cardiovascular event. Thus, their study design, analysis,
and purpose differ from ours.

\begin{table}
\begin{tabular}{l r}\label{tab:fram2wls}
Framingham study variable & WLS Variable\\
\hline
Sex & Sex\\
Quantitative total cholesterol & highchol2011 or highchol2004\\
Quantitative HDL cholesterol & None\\
Smoking & Columns 61 to 87 aim to quantify smoking \\
Diabetes & diabetes2004, diabetes2011, diabdiag2004, diabdiag2011\\
Age & Age\\
Systolic BP & highbp2004, highbp2011* \\
Treated for high blood pressure & None\\
\hline
\end{tabular}
\caption{Framingham study variables and their closest analogs in WLS. (* SBP not available, so we used reported "high BP".)}
\end{table}

Leek (2015)

\subsection*{References}\label{references}
\addcontentsline{toc}{subsection}{References}

D'Agostino, Ralph B, Ramachandran S Vasan, Michael J Pencina, Philip A
Wolf, Mark Cobain, Joseph M Massaro, and William B Kannel. 2008.
``General Cardiovascular Risk Profile for Use in Primary Care the
Framingham Heart Study.'' \emph{Circulation} 117 (6). Am Heart Assoc:
743--53.

Leek, Jeff. 2015. \emph{The Elements of Data Analytic Style}. Leanpub.

Mozaffarian, Dariush, Emelia J Benjamin, Alan S Go, Donna K Arnett,
Michael J Blaha, Mary Cushman, Sarah de Ferranti, et al. 2015.
``Executive Summary: Heart Disease and Stroke Statistics---2015 Update a
Report from the American Heart Association.'' \emph{Circulation} 131
(4). Am Heart Assoc: 434--41.

``Wisconsin Longitudinal Study''. 2015.
\url{http://www.ssc.wisc.edu/wlsresearch/}.

\end{document}
